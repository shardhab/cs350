\documentclass{article}

\usepackage{fancyhdr}
\usepackage{extramarks}
\usepackage{amsmath}
\usepackage{amsthm}
\usepackage{amsfonts}
\usepackage{tikz}
\usepackage[plain]{algorithm}
\usepackage{algpseudocode}
\usepackage{parskip}
\usepackage{enumerate}
\usetikzlibrary{automata,positioning}

%
% Basic Document Settings
%

\topmargin=-0.45in
\evensidemargin=0in
\oddsidemargin=0in
\textwidth=6.5in
\textheight=9.0in
\headsep=0.25in

\linespread{1.1}

\pagestyle{fancy}
\lhead{\hmwkAuthorName}
\chead{\hmwkClass\ (\hmwkClassInstructor): \hmwkTitle}
\rhead{\firstxmark}
\lfoot{\lastxmark}
\cfoot{\thepage}

\renewcommand\headrulewidth{0.4pt}
\renewcommand\footrulewidth{0.4pt}

\setlength\parindent{0pt}

%
% Create Problem Sections
%

\newcommand{\enterProblemHeader}[1]{
    \nobreak\extramarks{}{Problem \arabic{#1} continued on next page\ldots}\nobreak{}
    \nobreak\extramarks{Problem \arabic{#1} (continued)}{Problem \arabic{#1} continued on next page\ldots}\nobreak{}
}

\newcommand{\exitProblemHeader}[1]{
    \nobreak\extramarks{Problem \arabic{#1} (continued)}{Problem \arabic{#1} continued on next page\ldots}\nobreak{}
    \stepcounter{#1}
    \nobreak\extramarks{Problem \arabic{#1}}{}\nobreak{}
}

\setcounter{secnumdepth}{0}
\newcounter{partCounter}
\newcounter{homeworkProblemCounter}
\setcounter{homeworkProblemCounter}{1}
\nobreak\extramarks{Problem \arabic{homeworkProblemCounter}}{}\nobreak{}

%
% Homework Problem Environment
%
% This environment takes an optional argument. When given, it will adjust the
% problem counter. This is useful for when the problems given for your
% assignment aren't sequential. See the last 3 problems of this template for an
% example.
%
\newenvironment{homeworkProblem}[1][-1]{
    \ifnum#1>0
        \setcounter{homeworkProblemCounter}{#1}
    \fi
    \section{Problem \arabic{homeworkProblemCounter}}
    \setcounter{partCounter}{1}
    \enterProblemHeader{homeworkProblemCounter}
}{
    \exitProblemHeader{homeworkProblemCounter}
}

%
% Homework Details
%   - Title
%   - Due date
%   - Class
%   - Section/Time
%   - Instructor
%   - Author
%

\newcommand{\hmwkTitle}{Homework\ \#3}
\newcommand{\hmwkDueDate}{January 29, 2015}
\newcommand{\hmwkClass}{CS350 - Algorithms}
\newcommand{\hmwkClassInstructor}{for Professor Andrew Black}
\newcommand{\hmwkAuthorName}{Kristina Frye}

%
% Title Page
%

\title{
    \vspace{2in}
    \textmd{\textbf{\hmwkClass:\ \hmwkTitle}}\\
    \normalsize\vspace{0.1in}\small{Due\ on\ \hmwkDueDate}\\
    \vspace{0.1in}\large{\textit{\hmwkClassInstructor}}
    \vspace{3in}
}

\author{\textbf{\hmwkAuthorName}}
\date{}

\renewcommand{\part}[1]{\textbf{\large Part \Alph{partCounter}}\stepcounter{partCounter}\\}

%
% Various Helper Commands
%

% Useful for algorithms
\newcommand{\alg}[1]{\textsc{\bfseries \footnotesize #1}}

% For derivatives
\newcommand{\deriv}[1]{\frac{\mathrm{d}}{\mathrm{d}x} (#1)}

% For partial derivatives
\newcommand{\pderiv}[2]{\frac{\partial}{\partial #1} (#2)}

% Integral dx
\newcommand{\dx}{\mathrm{d}x}

% Alias for the Solution section header
\newcommand{\solution}{\textbf{\large Solution}}

% Probability commands: Expectation, Variance, Covariance, Bias
\newcommand{\E}{\mathrm{E}}
\newcommand{\Var}{\mathrm{Var}}
\newcommand{\Cov}{\mathrm{Cov}}
\newcommand{\Bias}{\mathrm{Bias}}

\begin{document}

\maketitle

\pagebreak

\begin{homeworkProblem}
Compute the following sums:
\begin{enumerate}[(a)]
\item
$\displaystyle\sum_{i=0}^{n+2}i(i + 1)$
\begin{align*}
\sum_{i=0}^{n+2}i(i + 1) &=\sum_{i=0}^{n+2}i^2 + i =
\sum_{i=0}^{n+2}i^2 + \sum_{i=0}^{n+2}i \\
&=\frac{n(n+1)(2n + 1)}{6} + (n+1)^2 + (n+2)^2 +
\frac{n(n+1)}{2} + (n+1) + (n+2)\\
&=\frac{2n^3 + 3n^2 + n}{6} + (n^2 + 2n + 1) + (n^2 + 4n + 4) +
\frac{n^2 + n}{2} + (n + 1) + (n + 2)\\
&=\frac{2n^3 + 3n^2 + n}{6} +
\frac{6(2n^2 + 6n + 5)}{6} + \frac{n^2 + n}{2} +
\frac{2(2n + 3)}{2}\\
&=\frac{2n^3 + 15n^2 + 37n + 30}{6} + \frac{n^2 + 5n + 6}{2}\\
&=\frac{2n^3 + 15n^2 + 37n + 30}{6} + \frac{3n^2 + 15n + 18}{6}\\
&=\frac{2n^3 + 18n^2 + 52n + 48}{6}
\end{align*}
\item
$\displaystyle\sum_{j=1}^{n}3^{j+2}$
\begin{align*}
\sum_{j=1}^{n}3^{j+2} &= \sum_{j=1}^{n}3^2 3^j = 3^2\sum_{j=1}^{n}3^j
= 3^2 \cdot \frac{3^{n+1}-1}{2} - 1
= 3^2 \cdot \frac{3^{n+1}-1-2}{2}
= 3^2 \cdot \frac{3^{n+1}-3}{2}
= \frac{3^{n+3} - 3^3}{2}\\
&= \frac{3^{n+3} - 27}{2}
\end{align*}

\item
$\displaystyle\sum_{i=1}^{n}\sum_{j=1}^{n}ij$
\begin{align*}
\sum_{i=1}^{n}\sum_{j=1}^{n}ij &=
\sum_{i=1}^{n}i\sum_{j=1}^{n}j =
\sum_{i=1}^{n}i \frac{n(n+1)}{2} =
\frac{n(n+1)}{2} \sum_{i=1}^{n}i =
\frac{n(n+1)}{2} \frac{n(n+1)}{2} = 
\frac{1}{4} n^2(n+1)^2\\
&= \frac{1}{4} n^2(n^2 + 2n + 1) =
\frac{1}{4} \left(n^4 + 2n^3 + n^2\right)
\end{align*}
\end{enumerate}
\end{homeworkProblem}
\pagebreak
\begin{homeworkProblem}
\begin{algorithmic}
\Procedure{Secret}{$A[0..n-1]$}\Comment{An array $A[0..n-1]$ of $n$ real numbers}
\State $minval \gets A[0]; maxval \gets A[0]$
\For{$i = 1 \text{ to } n-1$}
\If{$A[i] < minval$}
\State $minval \gets A[i]$
\EndIf
\If{$A[i] > maxval$}
\State $maxval \gets A[i]$
\EndIf
\EndFor
\State \textbf{return} $maxval - minval$
\EndProcedure
\end{algorithmic}
\begin{enumerate}[(a)]
\item What does this algorithm do? \textbf{A: } It computes the difference
between the maximum and minimum value of the array of numbers (the range).
\item What is its basic operation \textbf{A: } The comparison operator ($<$ and $>$).
\item How many times is the basic operation executed in the best and
worst cases? \textbf{A: } The best case is the same as the worse case.
For all cases, each element needs to be checked once; there is no way to
short circuit this algorithm (assuming there are no absolute min or max
values.) The comparison is performed twice for each loop from 
$i = 1$ to $n-1$. Therefore, it is performed $2 \cdot (n-1)$ times. 

\item What is the efficiency class of this algorithm? \textbf{A: }
$O(n)$ (linear)
\end{enumerate}
\end{homeworkProblem}

\begin{homeworkProblem}
Solve the following recurrence relations, using backwards substitution,
or by calculating the first few terms and generalizing.
\begin{enumerate}[(a)]
\item $x(n) = 4 x(n-1) \text{ for } n > 1, x(1) = 2$
\begin{align*}
x(n) &= 4 \cdot x(n-1) \\
\implies x(n-1) &= 4 \cdot x(n-2)\\
\implies x(n) &= 4 \cdot 4 \cdot x(n-2)\\
\implies x(n) &= 4^k \cdot x(n-k)\\
\text{Let }k=n-1. \text{ Then: }
x(n) &= 4^{n-1} \cdot x(n-(n-1))\\
&= 4^{n-1} \cdot x(1)\\
&= 4^{n-1} \cdot 2\\
\implies x(n) &= 2 \cdot 4^{n-1}
\end{align*}

\item $x(n) = x(n-1) + n \text{ for } n > 0, x(0) = 3$
\begin{align*}
x(n) &= x(n-1) + n\\
\implies x(n-1) &= x(n-2) + n-1\\
\implies x(n) &= x(n-2) + (n-1) + n\\
\implies x(n) &= x(n-k) + \sum_{k=0}^{n-1}n-k\\
\implies x(n) &= x(n-k) + \sum_{j=1}^{n}j\\
\implies x(n) &= x(n-k) + \frac{n(n+1)}{2}\\
\text{Let }k=n. \text{ Then: }
x(n) &= x(n-n) + \frac{n(n+1)}{2}\\
\implies x(n) &= 3 + \frac{n(n+1)}{2}\\
\implies x(n) &= \frac{n^2 + n + 6}{2}\\
\end{align*}
\item $x(n) = x(n/3) + 1 \text{ for } n > 1, x(1) = 1 \text{ (solve for }
n = 3k)$
\begin{align*}
x(n) &= x\left(\frac{n}{3}\right) + 1\\
\implies x\left(\frac{n}{3}\right) &= x\left(\frac{n}{9}\right) + 1\\
\implies x(n) &= x\left(\frac{n}{9}\right) + 1 + 1\\
\implies x(n) &= x\left(\frac{n}{3^k}\right) + k\\
\text{Let }n=3k. \text{ Then: }
x\left(3^k\right) &= x\left(\frac{3^k}{3^k}\right) + \log_3 3^k\\
&= 1 + \log_3 n\\
\implies x(n) &= 1 + \log_3 n \text{ where } n = 3^k
\end{align*}
\end{enumerate}
\end{homeworkProblem}

\begin{homeworkProblem}
Consider this algorithm for $min$; it recursively divides the input array
$A$ into two halves:
\begin{algorithm}
\begin{algorithmic}
\Procedure{Min2}{$A[l..r]$}
\If{$l=r$}
\State return $A[l]$
\EndIf
\State $mid \gets \left \lfloor\frac{l + r}{2}\right \rfloor$
\State $temp1 \gets Min2(A[l..mid])$
\State $temp2 \gets Min2(A[mid+1..r])$
\If{$temp1 \le temp2$}
\State return $temp1$
\Else
\State return $temp2$
\EndIf
\EndProcedure
\end{algorithmic}
\end{algorithm}
\begin{enumerate}[(a)]
\item What is this Algorithm’s basic operation?
\textbf{ A: }
The division operation $\frac{l + r}{2}$.
\item Set up a recurrence relation for $B(n)$, the number of times that
the basic operation is executed on an input array of size $n$, and
\textbf{ A: }
$B(n) = 2 \cdot B\left(\frac{n}{2}\right) + 1, B(1) = 0, n > 0$
\item solve it
\begin{align*}
B(n) &= 2 \cdot B\left(\frac{n}{2}\right) + 1\\
\implies B\left(\frac{n}{2}\right) &= 2 \cdot B\left(\frac{n}{4}\right) + 1\\
\implies B(n) &= 2 \cdot \left(2 \cdot B\left(\frac{n}{4}\right) + 
    1\right) + 1\\
\implies B(n) &= 2 \cdot 2 \cdot B\left(\frac{n}{4}\right) + 2 + 1\\
\implies B(n) &= 2^k B\left(\frac{n}{2^k}\right) + 
    \sum_{j=0}^{k-1}2^j\\
\implies B(n) &= 2^k B\left(\frac{n}{2^k}\right) + 2^{k} - 1\\
\text{Let } n = 2^k. \text{ Then: }
B(n) &= n \cdot B\left(\frac{n}{n}\right) + n - 1 \\
B(n) &= n - 1
\end{align*}
\end{enumerate}
\end{homeworkProblem}
\pagebreak
\begin{homeworkProblem}
    A network topology is a specification of how devices (computers,
    printers, etc.) are connected in a network. The figures below show two 
    different network topologies: star, and fully-connected.
    For each of these topologies, design a brute-force algorithm that 
    determines if a given graph has that topology. Assume $n > 3$, and that the 
    input is an adjacency matrix of 1 and 0 values. Do not assume that the 
    input matrix represents an undirected graph. Indicate time the efficiency 
    class of your algorithm — show your work.

    \textbf{Solution: Star Topology}
        \begin{algorithmic}
            \Function{IsStarTopology}{$A[0..n-1][0..n-1]$}
                \State $key = -1$ \Comment{1 op}
                \For{$i:=0 \text{ to } n-1$}
                    \State $count = 0$ \Comment{1 op}
                    \For{$j:=0 \text{ to } n-1$}
                        \State $count = count + A[i][j]$ 
                        \Comment{Count how many nodes are connected (1 op)}
                    \EndFor
                    \If{$count = n - 1$} 
                        \Comment{Central node identified (1 op)}
                        \If{$key = -1$} \Comment{1 op}
                            \State $key = i$ 
                        \Else
                            \State \textbf{return false} 
                            \Comment{There can be only one central node} 
                        \EndIf
                    \Else
                        \If{$count \neq 1$}
                            \Comment{All non-central nodes must have count of 
                              1 (1 op)}
                            \State \textbf{return false}
                        \EndIf
                    \EndIf 
                \EndFor
                \If{$key = -1$}
                    \Comment{There must be a central node (1 op)}
                    \State \textbf{return false}
                \EndIf
                \For{$i:=0 \text{ to } n-1$}
                    \Comment{Check for connection to identified central node}
                    \If{$A[i][key] = 0 \text{ AND } i \neq key$}
                        \Comment{Check each node is connected
                          to the central node (2 op)} 
                        \State \textbf{return false}
                    \EndIf
                \EndFor
                    \If{$A[key][key] \neq 0$}
                        \Comment{Check the central node is not connected
                          to itself (1 op)}
                        \State \textbf{return false}
                    \EndIf
                \State \textbf{return true}
            \EndFunction{}
    \end{algorithmic}
    Time efficency calculated assuming that additions and comparisons are
    counted as operations. Operations involved in for loops (additions and
    comparisons) are ignored since they will just add a constant to the
    total sum of operation calculations, are incidental to the algorithm,
    and won't affect the efficiency class. It is also assumed that $n-1$ is
    a know precalculated value. Worst-case operations are indicated in the
    comments shown in the algorithm. 
   
    \begin{align*} 
        ops(n) &= 3 + \sum_{i=0}^{n-1} \left(4 + \sum_{j=0}^{n-1} 1 \right) + 
        \sum_{i=0}^{n-1} 2 \\
        &= 3 + \sum_{i=0}^{n-1} \left(4 + n\right) + 2 \cdot n\\
        &= 3 + 2n + 4 \cdot \left(\sum_{i=0}^{n-1} 1\right) + 
           n \cdot \sum_{i=0}^{n-1} 1 \\
        &= 3 + 2n + 4n + n \cdot n\\
        &= 3 + 6n + n^2 \\
        &\in O(n^2) 
    \end{align*} 
  
    \textbf{Solution: Fully-Connected Topology}
    \begin{algorithmic}
        \Function{IsFullyConnected}{$A[0..n-1][0..n-1]$}
            \For{$i:=0 \text{ to } n-1$}
                \State $count = 0$ \Comment{1 op}
                \For{$j:=0 \text{ to } n-1$}
                    \State $count = count + A[i,j]$ 
                    \Comment{Count how many nodes are connected (1 op)}
                \EndFor
                    \Comment{There should be $n-1$ connected nodes and
                             a node should not be connected to itself}
                \If{$count \neq n - 1 \text{ OR } A[i,i] \neq 0$ }
                    \Comment{2 op}
                    \State \textbf{return false}
                \EndIf
            \EndFor
            \State \textbf{return true}
        \EndFunction
    \end{algorithmic}
    Time efficiency (same assumptions as before):
    \begin{align*}
        ops(n) &= \left( \sum_{i=0}^{n-1} 1 + \left( 
        \sum_{j=0}^{n-1} 1\right) + 2 \right)\\
        &= \sum_{i=0}^{n-1} 1 + n + 2 \\
        &= \sum_{i=0}^{n-1} 1 + n \cdot \sum_{i=0}^{n-1} 1 + 
           2 \cdot \sum_{i=0}^{n-1} 1\\
        &= n + n \cdot n + 2 \cdot n\\
        &= 3n + n^2\\
        &\in O(n^2) 
    \end{align*} 
\end{homeworkProblem}
\pagebreak
\begin{homeworkProblem}
A stack of fake coins There are $n$ stacks of $n$ identical-looking coins. 
All of the coins in one of these stacks are counterfeit, while all the coins 
in the other stacks are genuine. Every genuine coin weighs 10 grams; every fake 
weighs 11 grams. You have an analytical scale that can determine the exact 
weight of any number of coins.
\begin{enumerate}[(a)]
    \item Devise a brute-force algorithm to identify the stack with the fake 
          coins, and 
        \begin{algorithmic}
            \Function{IndexOfFakeCoins}{$A[0..n-1]$}
                \Comment{Array of $n$ stacks of $n$ coins}
                \For{$i:=0 \text{ to } n-1$}
                    \If{$weight = 11n$}
                        \State return $i$
                    \EndIf
                \EndFor
             \EndFunction
        \end{algorithmic}
    \item determine its worst-case efficiency class:\\
    Basic operation = weight function. Worst case: last stack\\
    \[
        ops(n) = \sum_{i=0}^{n-1} 1 = n \in O(n)
    \]
\end{enumerate}
\end{homeworkProblem}

\begin{homeworkProblem}
Alternating disks You have a row of 2n disks of two colors, $n$ dark and $n$ 
light. They alternate: dark, light, dark, light, and so on. You want to get 
all the dark disks to the right-hand end, and all the light disks to the 
left-hand end. The only moves you are allowed to make are those that 
interchange the positions of two neighboring disks. Design an algorithm for 
solving this puzzle and determine the number of moves it makes.

\textbf{Solution:}
\begin{algorithmic}
    \Function{SortDisks}{$A[0..2n-1]$}
        \For{$i:=0 \text{ to } n - 1$}
            \Comment{Array should be sorted after halfway through}
            \For{$j:=0 \text{ to } n - 1 - i$}
                \Comment{Only need to check every other index}
                \If{$A[2j + i] = D \text{ AND } A[2j + i + 1] = L$}
                    \State $flip(A[2j + i], A[2j + i + 1])$
                \EndIf 
            \EndFor
        \EndFor
    \EndFunction
\end{algorithmic}
Basic operation: flip
\begin{align*}
    ops(n) &= \sum_{i=0}^{n-1} \sum_{j=0}^{n - i - 1} 1\\
    &= \sum_{i=0}^{n-1} n - 1 - i + 1 = \sum_{i=0}^{n-1} n - i\\
    &= \sum_{j=1}^{n} j = \frac{n(n+1)}{2}\\
    &\in O(n^2)
\end{align*}
\end{homeworkProblem}
\end{document}
