\documentclass[11pt]{article}
\usepackage{amsmath,amssymb,amsthm,tabu,enumerate,tikz}
\usepackage{algorithm}
\usepackage{algpseudocode}
\usetikzlibrary{automata,positioning}
\usepackage[margin=1in]{geometry}
\renewcommand{\thealgorithm}{}
\begin{document}

{\raggedleft Kristina Frye \par}
{\raggedleft CS 350 \par}
{\raggedleft January 29, 2015 \par}
{\raggedleft Assignment 3 \par} 
\noindent\textbf{Problem 1} \\
Compute the following sums:
\begin{enumerate}[(a)]
\item
$\displaystyle\sum_{i=0}^{n+2}i(i + 1)$
\begin{align*}
\sum_{i=0}^{n+2}i(i + 1) &=\sum_{i=0}^{n+2}i^2 + i =
\sum_{i=0}^{n+2}i^2 + \sum_{i=0}^{n+2}i \\
&=\frac{n(n+1)(2n + 1)}{6} + (n+1)^2 + (n+2)^2 + 
\frac{n(n+1)}{2} + (n+1) + (n+2)\\
&=\frac{2n^3 + 3n^2 + n}{6} + (n^2 + 2n + 1) + (n^2 + 4n + 4) +
\frac{n^2 + n}{2} + (n + 1) + (n + 2)\\
&=\frac{2n^3 + 3n^2 + n}{6} + 
\frac{6(2n^2 + 6n + 5)}{6} + \frac{n^2 + n}{2} +
\frac{2(2n + 3)}{2}\\
&=\frac{2n^3 + 15n^2 + 37n + 30}{6} + \frac{n^2 + 5n + 6}{2}\\
&=\frac{2n^3 + 15n^2 + 37n + 30}{6} + \frac{3n^2 + 15n + 18}{6}\\
&=\frac{2n^3 + 18n^2 + 52n + 48}{6}
\end{align*}
\item
$\displaystyle\sum_{j=1}^{n}3^{j+2} = 
\sum_{j=1}^{n}3^2 3^j = 3^2\sum_{j=1}^{n}3^j
= 3^2 \cdot \frac{3^{n+1}-1}{2}
= \frac{3^{n+3} - 3^2}{2}
= \frac{3^{n+3} - 9}{2}$

\item
$\displaystyle\sum_{i=1}^{n}\sum_{j=1}^{n}ij = 
\sum_{i=1}^{n}i\sum_{j=1}^{n}j =
\sum_{i=1}^{n}i \frac{n(n+1)}{2} =
\frac{n(n+1)}{2} \sum_{i=1}^{n}i =
\frac{n(n+1)}{2} \frac{n(n+1)}{2} =
\frac{1}{4} n^2(n+1)^2$ 
\end{enumerate}
\noindent\textbf{Problem 2}\\ 
Consider the following algorithm:
\begin{algorithm}
\begin{algorithmic}
\Procedure{Secret}{$A[0..n-1]$}\Comment{An array $A[0..n-1]$ of $n$ real numbers}
\State $minval \gets A[0]; maxval \gets A[0]$
\For{$i = 1 \text{ to } n-1$}
\If{$A[i] < minval$}
\State $minval \gets A[i]$
\EndIf
\If{$A[i] > maxval$}
\State $maxval \gets A[i]$
\EndIf
\EndFor
\State \textbf{return} $maxval - minval$
\EndProcedure
\end{algorithmic}
\end{algorithm}
\begin{enumerate}[(a)]
\item What does this algorithm do? \textbf{A: } It computes the difference
between the maximum and minimum value of the array of numbers.
\item What is its basic operation \textbf{A: } The comparison operator ($<$ and $>$).
\item How many times is the basic operation executed in the best and 
worst cases? \textbf{A: } The best case is the same as the worse case. 
For all cases, each element needs to be checked once; there is no way to 
short circuit this algorithm, assuming there are no absolute min or max 
values. Therefore $2n$ operations will always need to be performed in 
order to determine the min and max values of the entire array. 
\item What is the efficiency class of this algorithm? \textbf{A: } 
$O(n)$ (linear)
\end{enumerate}
\noindent\textbf{Problem 3:}\\
Solve the following recurrence relations, using backwards substitution, 
or by calculating the first few terms and generalizing.
\begin{enumerate}[(a)]
\item $x(n) = 4 x(n-1) \text{ for } n > 1, x(1) = 2$
\begin{align*}
x(n) &= 4 \cdot x(n-1) \\
\implies x(n-1) &= 4 \cdot x(n-2)\\
\implies x(n) &= 4 \cdot 4 \cdot x(n-2)\\
\implies x(n) &= 4^k \cdot x(n-k)\\
\text{Let }k=n-1. \text{ Then: } 
x(n) &= 4^{n-1} \cdot x(n-(n-1))\\
&= 4^{n-1} \cdot x(1)\\
&= 4^{n-1} \cdot 2\\
\implies x(n) &= 2 \cdot 4^{n-1}
\end{align*}

\item $x(n) = x(n-1) + n \text{ for } n > 0, x(0) = 3$
\begin{align*}
x(n) &= x(n-1) + n\\
\implies x(n-1) &= x(n-2) + n-1\\
\implies x(n) &= x(n-2) + (n-1) + n\\
\implies x(n) &= x(n-k) + \sum_{k=0}^{n-1}n-k\\
\implies x(n) &= x(n-k) + \sum_{j=1}^{n}j\\
\implies x(n) &= x(n-k) + \frac{n(n+1)}{2}\\
\text{Let }k=n. \text{ Then: } 
x(n) &= x(n-n) + \frac{n(n+1)}{2}\\
\implies x(n) &= 3 + \frac{n(n+1)}{2}\\
\end{align*}

\item $x(n) = x(n/3) + 1 \text{ for } n > 1, x(1) = 1 \text{ (solve for } 
n = 3k)$
\begin{align*}
x(n) &= x\left(\frac{n}{3}\right) + 1\\
\implies x\left(\frac{n}{3}\right) &= x\left(\frac{n}{9}\right) + 1\\
\implies x(n) &= x\left(\frac{n}{9}\right) + 1 + 1\\
\implies x(n) &= x\left(\frac{n}{3^k}\right) + k\\
\text{Let }n=3k. \text{ Then: }
x\left(3^k\right) &= x\left(\frac{3^k}{3^k}\right) + \log_3 3^k\\
&= 1 + \log_3 n\\
\implies x(n) &= 1 + \log_3 n \text{ where } n = 3^k 
\end{align*}
\end{enumerate}
\noindent\textbf{Problem 4} \\
Consider this algorithm for $min$; it recursively divides the input array 
$A$ into two halves:
\begin{algorithm}
\begin{algorithmic}
\Procedure{Min2}{$A[l..r]$}
\If{$l=r$}
\State return $A[l]$
\EndIf
\State $mid \gets \left \lfloor\frac{l + r}{2}\right \rfloor$
\State $temp1 \gets Min2(A[l..mid])$
\State $temp2 \gets Min2(A[mid+1..r])$
\If{$temp1 \le temp2$}
\State return $temp1$
\Else
\State return $temp2$
\EndIf
\EndProcedure
\end{algorithmic}
\end{algorithm}
\begin{enumerate}[(a)]
\item What is this Algorithm’s basic operation? 
\textbf{ A: } 
The division operation $\frac{l + r}{2}$.
\item Set up a recurrence relation for $B(n)$, the number of times that 
the basic operation is executed on an input array of size $n$, and

\textbf{ A: }
$B(n) = 2 \cdot B\left(\frac{n}{2})\right) + 1, B(1) = 0, n > 0$
\item solve it
\begin{align*}
B(n) &= 2 \cdot B\left(\frac{n}{2}\right) + 1\\
\implies B\left(\frac{n}{2}\right) &= 2 \cdot B\left(\frac{n}{4}\right) + 1\\
\implies B(n) &= 2 \cdot 2 \cdot B\left(\frac{n}{4}\right) + 1 + 1\\
\implies B(n) &= 2^k B\left(\frac{n}{2^k}\right) + k\\
\text{Let } n = 2^k. \text{ Then: } 
B(n) &= n \cdot B\left(\frac{n}{n}\right) + \log_2n\\
B(n) &= 1 + \log_2n
\end{align*}
\end{enumerate}
\noindent\textbf{Proof:}\\
Let: 
\begin{align*}
t_1 &= a_kn^k, \\
t_2 &= a_{k-1}n^{k-1}, \\
t_3 &= a_{k-2}n^{k-2}, \\ 
...\\
t_k &= a_0n^0
\end{align*}
Then:
\begin{align*}
t_1 &\in \Theta(n^k),\\
t_2 &\in \Theta(n^{k-1}),\\
t_3 &\in \Theta(n^{k-2}),\\
...\\
t_k &\in \Theta(n^0)
\end{align*}
by the definition of $\Theta$.
But $(t_1 + t_2 + t_3 + ... + t_k) \in \Theta(\max\{ n^k, n^{k-1}, n^{k-2},
..., n^0\})$ by the theorem on page $56$ in the book and 
$\max\{n^k, n^{k-1}, n^{k-2}, ..., n^0\}) = n^k.$ Therefore
$p(n) = a_kn^k + a_{k-1}n^{k-1} + a_{k-2}n^{k-2} + a_{k-3}n^{k-3} + ... +
a_0 \in \Theta(n^k)$ which means that every polynomial $\in \Theta(n^k) \qed$ 
\\
\\
\noindent\textbf{Problem 4} \\
Levitin mentions, in section 2, that one can check whether all elements of 
an array are distinct by a two-part algorithm that first sorts the array, and
then scans through the array looking at adjacent elements. If the sorting is
done by an algorithm with the time efficiency in $\Theta(n \log n)$, what will
be the time efficiency class of the entire algorithm? Explain why.\\ 
\\
\noindent\textbf{Answer}\\
The time efficiency class of the entire algorithm will also be $\Theta(n \log
n)$. In a two part algorithm, the efficiency of the entire algorithm is the 
addition of the second part of the algorithm to the first part of the 
algorithm. Let the first part of the algorithm be $t_1(n) \in \Theta(n \log
n)$. Let the second part of the algorithm (scanning through the array looking
at adjacent elements) be $t_2(n)$. $t_2(n) \in \Theta(n)$
because each member of the array will need to be visited once (after they
have been sorted) to confirm that each element is unique. This is linear
time $(\Theta(n))$. The time efficiency class of the entire algorithm is
$\Theta(\max \{n \log n, n\}) = \Theta(n \log n)$ by the theorem on page $56$
in the book. \\
\\ 
\noindent\textbf{Problem 5}\\
The range of a finite nonempty set of $n$ real numbers $S$ is defined as the
difference between the largest and smallest elements of $S$. For each
representation of $S$ given below, describe in English an algorithm to compute
the range. Indicate the time efficiency classes of these algorithms using the
most appropriate notation $(O, \Theta, \text{or } \Omega)$. 
\begin{enumerate}[(a)]
\item An unsorted array: Set local variables max and min equal to the first
element of the array. Step through each element of the array. If the value
is greater than max, set max equal to the value. If the value is less than
min, set min equal to the value. After each element has been visited,
subtract min from max and return the resulting value. This is $\Theta(n)$
since each element of the array is visited exactly once. 
\item A sorted array: Subtract the first element of the array (the smallest
number) from the last element of the array (the largest number). This is
constant time: $\Theta(1)$.
\item A sorted singly-linked list: Set the first element of the list equal
to a local variable min. Traverse through the list until you get to the last
element. Set the value of the last element equal to local variable max.
Subtract min from max and return the result. This is linear time: $\Theta(n)$
since each element must be visited once.
\item A binary search tree: Go to the first element of the binary search
tree by choosing the left branch when it is exists. If no left branch
exists, choose the right. Continue until a leaf is reached. This leaf will
contain the minumum value. Store into a local variable.
Starting from the root node, traverse the tree, choosing the right branch
when it is available. If no right branch is available, choose the left 
branch. Continue until a leaf is reached. This leaf will contain 
the maximum value. Subtract the minimum from this value and return the
result. The efficiency class is dependent upon the structure of the
binary search tree. When finding the min and max values, the worst case is 
$O(n)$ when the tree is set up similar to a linked list with the desired
value at the end of a long chain. The efficiency class could be
$\Theta(\log n)$ if the tree is balanced. Therefore, the proper categorization 
is the linear efficiency class: $O(n) + O(n) = O(n)$. 
\end{enumerate}
\end{document}
