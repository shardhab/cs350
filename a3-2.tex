\documentclass{article}

\usepackage{fancyhdr}
\usepackage{extramarks}
\usepackage{amsmath}
\usepackage{amsthm}
\usepackage{amsfonts}
\usepackage{tikz}
\usepackage[plain]{algorithm}
\usepackage{algpseudocode}
\usepackage{parskip}
\usepackage{enumerate}
\usetikzlibrary{automata,positioning}

%
% Basic Document Settings
%

\topmargin=-0.45in
\evensidemargin=0in
\oddsidemargin=0in
\textwidth=6.5in
\textheight=9.0in
\headsep=0.25in

\linespread{1.1}

\pagestyle{fancy}
\lhead{\hmwkAuthorName}
\chead{\hmwkClass\ (\hmwkClassInstructor): \hmwkTitle}
\rhead{\firstxmark}
\lfoot{\lastxmark}
\cfoot{\thepage}

\renewcommand\headrulewidth{0.4pt}
\renewcommand\footrulewidth{0.4pt}

\setlength\parindent{0pt}

%
% Create Problem Sections
%

\newcommand{\enterProblemHeader}[1]{
    \nobreak\extramarks{}{Problem \arabic{#1} continued on next page\ldots}\nobreak{}
    \nobreak\extramarks{Problem \arabic{#1} (continued)}{Problem \arabic{#1} continued on next page\ldots}\nobreak{}
}

\newcommand{\exitProblemHeader}[1]{
    \nobreak\extramarks{Problem \arabic{#1} (continued)}{Problem \arabic{#1} continued on next page\ldots}\nobreak{}
    \stepcounter{#1}
    \nobreak\extramarks{Problem \arabic{#1}}{}\nobreak{}
}

\setcounter{secnumdepth}{0}
\newcounter{partCounter}
\newcounter{homeworkProblemCounter}
\setcounter{homeworkProblemCounter}{1}
\nobreak\extramarks{Problem \arabic{homeworkProblemCounter}}{}\nobreak{}

%
% Homework Problem Environment
%
% This environment takes an optional argument. When given, it will adjust the
% problem counter. This is useful for when the problems given for your
% assignment aren't sequential. See the last 3 problems of this template for an
% example.
%
\newenvironment{homeworkProblem}[1][-1]{
    \ifnum#1>0
        \setcounter{homeworkProblemCounter}{#1}
    \fi
    \section{Problem \arabic{homeworkProblemCounter}}
    \setcounter{partCounter}{1}
    \enterProblemHeader{homeworkProblemCounter}
}{
    \exitProblemHeader{homeworkProblemCounter}
}

%
% Homework Details
%   - Title
%   - Due date
%   - Class
%   - Section/Time
%   - Instructor
%   - Author
%

\newcommand{\hmwkTitle}{Homework\ \#3}
\newcommand{\hmwkDueDate}{January 29, 2015}
\newcommand{\hmwkClass}{CS350 - Algorithms}
\newcommand{\hmwkClassInstructor}{Professor Andrew Black}
\newcommand{\hmwkAuthorName}{Kristina Frye}

%
% Title Page
%

\title{
    \vspace{2in}
    \textmd{\textbf{\hmwkClass:\ \hmwkTitle}}\\
    \normalsize\vspace{0.1in}\small{Due\ on\ \hmwkDueDate\ at 3:10pm}\\
    \vspace{0.1in}\large{\textit{\hmwkClassInstructor}}
    \vspace{3in}
}

\author{\textbf{\hmwkAuthorName}}
\date{}

\renewcommand{\part}[1]{\textbf{\large Part \Alph{partCounter}}\stepcounter{partCounter}\\}

%
% Various Helper Commands
%

% Useful for algorithms
\newcommand{\alg}[1]{\textsc{\bfseries \footnotesize #1}}

% For derivatives
\newcommand{\deriv}[1]{\frac{\mathrm{d}}{\mathrm{d}x} (#1)}

% For partial derivatives
\newcommand{\pderiv}[2]{\frac{\partial}{\partial #1} (#2)}

% Integral dx
\newcommand{\dx}{\mathrm{d}x}

% Alias for the Solution section header
\newcommand{\solution}{\textbf{\large Solution}}

% Probability commands: Expectation, Variance, Covariance, Bias
\newcommand{\E}{\mathrm{E}}
\newcommand{\Var}{\mathrm{Var}}
\newcommand{\Cov}{\mathrm{Cov}}
\newcommand{\Bias}{\mathrm{Bias}}

\begin{document}

\maketitle

\pagebreak

\begin{homeworkProblem}
    A network topology is a specification of how devices (computers,
    printers, etc.) are connected in a network. The figures below show two 
    different network topologies: star, and fully-connected.
    For each of these topologies, design a brute-force algorithm that 
    determines if a given graph has that topology. Assume $n > 3$, and that the 
    input is an adjacency matrix of 1 and 0 values. Do not assume that the 
    input matrix represents an undirected graph. Indicate time the efficiency 
    class of your algorithm — show your work.

    \textbf{Solution: Star Topology}
        \begin{algorithmic}
            \Function{IsStarTopology}{$A[0..n][0..n]$}
                \State $key = -1$ \Comment{1 op}
                \For{$i:=0 \text{ to } n$}
                    \State $count = 0$ \Comment{1 op}
                    \For{$j:=0 \text{ to } n$}
                        \State $count = count + 1$  
                        \Comment{Count how many nodes are connected (1 op)}
                    \EndFor
                    \If{$count = n - 1$} 
                        \Comment{Central node identified (1 op)}
                        \If{$key = -1$} \Comment{1 op}
                            \State $key = i$  \Comment{1 op}
                        \Else
                            \State \textbf{return false} 
                            \Comment{There can be only one central node} 
                        \EndIf
                    \Else
                        \If{$count \neq 1$}
                            \Comment{All non-central nodes must have count of 1}
                            \State \textbf{return false}
                        \EndIf
                    \EndIf 
                \EndFor
                \If{$key = -1$}
                    \Comment{There must be a central node (1 op)}
                    \State \textbf{return false}
                \EndIf
                \For{$i:=0 \text{ to } n$}
                    \Comment{Check for connection to identified central node}
                    \If{A[i][key] = 0 \text{ AND } $i \neq key$}
                        \Comment{2 op}
                        \State \textbf{return false}
                    \EndIf
                \EndFor
                \State \textbf{return true}
            \EndFunction{}
    \end{algorithmic}
    Time efficency calculated assuming that additions and comparisons are
    counted as operations. Operations involved in for loops (additions and
    comparisons) are ignored since they will just add constants to the
    total sum of operation calculations and won't affect the efficiency 
    class. Worst-case operations are indicated in the comments shown in the 
    algorithm. 
   
    \begin{align*} 
        ops(n) &= 2 + \left( \sum_{i=0}^{n-1} 4 + \sum_{j=0}^{n-1} 1 \right) + 
        \sum_{i=0}^{n-1} 2 \\
        &= 2 + \left(\sum_{i=0}^{n-1} 4 + n\right) + 2 \cdot n\\
        &= 2 + 4 \cdot \left(\sum_{i=0}^{n-1} 1\right) + 
           n \cdot \left(\sum_{i=0}^{n-1} 1 \right) + 2 \cdot n\\
        &= 2 + 4n + n \cdot n + 2n\\
        &= 2 + 6n + n^2 \\
        &\in O(n^2) 
    \end{align*} 
  
    \textbf{Solution: Fully-Connected Topology}
    \begin{algorithmic}
        \Function{IsFullyConnected}{$A[0..n][0..n]$}
            \For{$i:=0 \text{ to } n$}
                \State $count = 0$ \Comment{1 op}
                \For{$j:=0 \text{ to } n$}
                    \State $count = count + A[i,j]$ 
                    \Comment{Count how many nodes are connected (1 op)}
                \EndFor
                    \Comment{There should be $n-1$ connected nodes and
                             a node should not be connected to itself}
                \If{$count \neq n - 1 \text{ OR } A[i,i] \neq 0$ }
                    \Comment{2 op}
                    \State \textbf{return false}
                \EndIf
            \EndFor
            \State \textbf{return true}
        \EndFunction
    \end{algorithmic}
    Time efficiency (same assumptions as before):
    \begin{align*}
        ops(n) &= \left( \sum_{i=0}^{n-1} 1 + \left( 
        \sum_{j=0}^{n-1} 1\right) + 2 \right)\\
        &= \sum_{i=0}^{n-1} 1 + n + 2 \\
        &= \sum_{i=0}^{n-1} 1 + n \cdot \sum_{i=0}^{n-1} 1 + 
           2 \cdot \sum_{i=0}^{n-1} 1\\
        &= n + n \cdot n + 2 \cdot n\\
        &= 3n + n^2\\
        &\in O(n^2) 
    \end{align*} 
\end{homeworkProblem}

\begin{homeworkProblem}
A stack of fake coins There are $n$ stacks of $n$ identical-looking coins. 
All of the coins in one of these stacks are counterfeit, while all the coins 
in the other stacks are genuine. Every genuine coin weighs 10 grams; every fake 
weighs 11 grams. You have an analytical scale that can determine the exact 
weight of any number of coins.
\begin{enumerate}[(a)]
    \item Devise a brute-force algorithm to identify the stack with the fake 
          coins, and 
        \begin{algorithmic}
            \Function{IndexOfFakeCoins}{$A[0..n-1]$}
                \Comment{Array of $n$ stacks of $n$ coins}
                \For{$i:=0 \text{ to } n-1$}
                    \If{$weight = 11n$}
                        \State return $i$
                    \EndIf
                \EndFor
             \EndFunction
        \end{algorithmic}
    \item determine its worst-case efficiency class:\\
    Basic operation = weight function. Worst case: last stack\\
    \[
        ops(n) = \sum_{i=0}^{n-1} 1 = n \in O(n)
    \]
\end{enumerate}
\end{homeworkProblem}

\begin{homeworkProblem}
Alternating disks You have a row of 2n disks of two colors, $n$ dark and $n$ 
light. They alternate: dark, light, dark, light, and so on. You want to get 
all the dark disks to the right-hand end, and all the light disks to the 
left-hand end. The only moves you are allowed to make are those that 
interchange the positions of two neighboring disks. Design an algorithm for 
solving this puzzle and determine the number of moves it makes.

\textbf{Solution:}
\begin{algorithmic}
    \Function{SortDisks}{$A[0..2n-1]$}
        \For{$i:=0 \text{ to } n - 1$}
            \Comment{Array should be sorted after halfway through}
            \For{$j:=i \text{ to } n - 1 - i$}
                \Comment{Only need to check every other index}
                \If{$A[i + 2j] = D \text{ AND } A[i + 2j + 1] = L$}
                    \State $flip(A[i + 2j], A[i + 2j + 1])$
                \EndIf 
            \EndFor
        \EndFor
    \EndFunction
\end{algorithmic}
Basic operation: flip
\begin{align*}
    ops(n) &= \sum_{i=0}^{n-1} \sum_{j=0}^{n - 1 - i} 1\\
    &= \sum_{i=0}^{n-1} n - 1 - i + 1 = \sum_{i=0}^{n-1} n - i\\
    &= \sum_{j=1}^{n} j = \frac{n(n+1)}{2}\\
    &\in O(n^2)
\end{align*}
\end{homeworkProblem}
\end{document}
